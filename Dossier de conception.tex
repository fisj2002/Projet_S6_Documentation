% !TEX TS-program = pdflatexmk
% !TEX encoding = UTF-8 Unicode
\documentclass[12pt,titlepage=true,letterpaper,pointlessnumbers,headings=normal,captions=nooneline]{scrartcl}

\usepackage[french]{babel}
\selectlanguage{french}
\usepackage[T1]{fontenc}
\usepackage[utf8]{inputenc}

\usepackage[automark,markcase=upper]{scrlayer-scrpage}
\usepackage[protrusion=true,expansion]{microtype}
\usepackage{amsmath}
\usepackage{graphicx}
\usepackage{array,booktabs,longtable,tabularx}
\usepackage{arydshln}
\usepackage{dashrule}
\usepackage[squaren,cdot,textstyle]{SIunits}
\usepackage{ccicons}
\usepackage{enumerate}
\usepackage{url}

\usepackage{xcolor}
\definecolor{myred}{rgb}{.7,.13,.13}
\definecolor{myyellow}{rgb}{1,1,.79}
\definecolor{monbleu}{rgb}{.2,.4,.8}

\definecolor{vertUdS}{RGB}{0,119,73} % Pantone vert 3415
\definecolor{bleuTetrad}{RGB}{0,45,117} % Pantone Tetrad
\definecolor{rougTetrad}{RGB}{117,0,45} % Pantone Tetrad
\definecolor{oranTetrad}{RGB}{117,72,0} % Pantone Tetrad

% Sérif: lining figures in math, osf in text
\usepackage[scaled=.98,sups,osf]{XCharter}% 
% Sans sérif:
\usepackage[scaled=0.90]{helvet}%
% Mono:
\usepackage[scaled=0.85]{beramono}
% Math:
\usepackage[charter,bigdelims,vvarbb,scaled=1.07]{newtxmath}
\usepackage[cal=boondoxo]{mathalfa} % doc à regarder: intéressant.

% Math: autre \usepackage[osf,sc]{mathpazo}

\linespread{1.1}

% À voir:
%\DeclareFixedFont{\FonteTitre}{T1}{phv}{b}{n}{48pt}
\usepackage{pifont}

% Dimension de la page:
\setlength{\textheight}{672pt}
\setlength{\footskip}{24pt}

% Réglage de certaine apparence:
% Caption
\setkomafont{caption}{\sffamily}
\setkomafont{captionlabel}{\bfseries}
\setcapindent{0em}
% titre: section et autre
\addtokomafont{section}{%
	\color{rougTetrad}%
	\usesizeofkomafont{subsection}%
	}%
% Entête et pied de page:
\ihead{}
\chead{}
\ohead{\sffamily\scriptsize\scshape\textls[150]{\headmark}}
\ifoot{}
\cfoot{}
\ofoot{\normalfont\sffamily\small\thepage}
\renewcommand*{\sectionmarkformat}{}% pas de numéro de section

% Pas d'indentation
\setlength{\parindent}{0pt}

\begin{document}
%%% Titre
\thispagestyle{empty}
\noindent
\begin{center}
{\scshape \textls[150]{Université de Sherbrooke}\\[4pt] \textls[75]{Département de génie électrique et de génie informatique}}

\vspace{3cm}
\textsf{\large Session S6e 2018}

\vspace{.5cm}
\textsf{\Large Dossier de conception du projet de session}

\vspace{.3cm}
\textsf{\large Équipe P3}

\vspace{.25cm}
\textsf{\footnotesize remis le 20 septembre 2018}

\end{center}
\vfill

\noindent
\rule{\linewidth}{.8pt}

\noindent
Abibsi, Amazigh \hfill abia2601\\
Bouchard, Raphael \hfill bour2326\\
Carignan, Charles \hfill carc2923\\
Courtois, Sébastien \hfill cous2509\\
Denommée, Édouard \hfill dene2303\\
Fisher, Jeffrey \hfill fisj2002\\
Houde, Marc-Antoine \hfill houm3306\\


\newpage
\tableofcontents

\newpage
\listoffigures

\newpage
\listoftables


\newpage
\section{Contexte du projet et du marché}

\section{Cahier de charges fonctionnelles}

\section{Requis au niveau RF}

\section{Prototype/preuve de concept de S6}

\section{Outils de gestion}


\end{document}